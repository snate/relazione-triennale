% !TEX encoding = UTF-8
% !TEX TS-program = pdflatex
% !TEX root = ../tesi.tex
% !TEX spellcheck = it-IT

%**************************************************************
\chapter{Strategia aziendale}
\label{cap:strat-aziend}
%**************************************************************

\intro{Questo capitolo serve per descrivere come l'azienda percepisce le
attività di stage e come le gestisce, parlando in particolare dello stage a
cui la relazione finale fa riferimento.}\\

%**************************************************************
\section{Il progetto}\label{sec:strat-prog}

Il progetto di stage era rivolto all'estensione dell'area Scrum di JPA.

Tale area, nata grazie ad un'altra attività di stage, viene utilizzata
quotidianamente dal team in cui sono stato inserito.

Il mio compito durante l'attività di stage è stato aggiungere funzionalità
che rendessero più soddisfacente l'uso dell'area Scrum di JPA. Questo è stato
possibile riconoscendo alcuni limiti che ad inizio stage JPA presentava:

\begin{itemize}
\item Per poter essere utilizzato da team che adottano la metodologia Scrum,
  vi era la necessità di studiare le differenze tra quanto previsto da Scrum e
  quanto offerto attualmente da questa area di JPA;
\item La decomposizione dei task più complessi era frequentemente fonte di
  problemi, siccome rendeva poco comprensibile il visualizzatore di sprint per
  le seguenti cause:
  \begin{itemize}
  \item Il meccanismo già presente per dire che un task è padre di un altro non
    consentiva una visione chiara della relazione, che non veniva evidenziata
    nel visualizzatore di sprint;
  \item Sia esplicitando la relazione padre-figlio tra diversi task che
    evitando di farlo, per suddividere un task in più sotto-task dovevano
    essere create \textbf{molte} attività, che compromettevano la leggibilità
    del visualizzatore di sprint.
  \end{itemize}
\item Nonostante uno dei principi fondamentali di Scrum sia l'\emph{Inspection}
  dei propri processi, non vi era nessuno strumento a sostegno della verifica
  e il controllo dei propri processi;
\item Ai clienti che fanno uso delle aree di modellazione processi di JPA
  potrebbe servire un collegamento tra queste e l'area Scrum;
\item L'area Scrum di JPA prevedeva esclusivamente il \emph{pull model} come
  modalità per ottenere informazioni: questo problema incideva chiaramente
  sull'efficienza del team, dal momento che non era possibile ottenere
  informazioni se non per controllo manuale da parte degli utenti.
\end{itemize}

In tabella \ref{tab:piano-di-lavoro} è riportato il Piano di Lavoro previsto
inizialmente per le otto settimane di stage: \\

\begin{tabular}{| c | p{10cm} |}

\hline
\textbf{Settimana} & \textbf{Lavoro previsto} \\
\hline
1 & Ricerca, studio e documentazione sulla metodologia scrum. Introduzione
    ai linguaggi di sviluppo. Analisi dello stato attuale del prodotto e
    confronto con la parte teorica. \\
\hline
2 & Analisi iniziale sulla gestione delle checklist legate ai task.
    Sviluppo modulo anagrafico di gestione checklist. Documentazione modulo
    anagrafico di gestione checklist. \\
\hline
3 & Sviluppo gestione checklist legate ai task. Documentazione relativa. \\
\hline
4 & Analisi su come produrre il burn-down chart con i dati presenti a
    sistema. Analisi ed eventuale sviluppi sui dati mancanti e necessari
    alla produzione del burn-down chart. Inizio sviluppo modulo di
    configurazione chart. \\
\hline
5 & Eventuale completamento del modulo di configurazione chart. Sviluppo e
    documentazione del burn-down chart. \\
\hline
6 & Completamento documentazione e tesi con obiettivi raggiunti e spunti
    per possibili altri sviluppi futuri. Analisi, sviluppo e
    documentazione gestione avvio istanza da task scrum. \\
\hline
7 & Analisi, sviluppo e documentazione sul collegamento tra modulo scrum e
    modulo di notifica. \\
\hline
8 & Analisi, sviluppo e documentazione sul collegamento tra modulo scrum e
    modulo forum. \\
\hline
\end{tabular}
\captionof{table}{Piano di Lavoro}
\label{tab:piano-di-lavoro}

%**************************************************************
\subsection{Il progetto visto dall'azienda}\label{sec:strat-ob-prod}

Questo stage avrebbe portato all'azienda diversi vantaggi, distinti in
obiettivi (produttivi) minimi e massimi. Sia che fossero minimi che massimi,
entrambi i tipi di obiettivi prevedevano la stesura di documentazione per
quanto sarebbe stato prodotto durante l'attività di stage. \\

\begin{tabular}{| p{6cm} | p{6cm} |}

\hline
\textbf{Obiettivi minimi} & \textbf{Obiettivi massimi} \\
\hline
Confronto tra stato attuale di JPA e Scrum ``puro'' &
Avvio di un'istanza di processo da un task \\
\hline
Sviluppo di un modulo di checklist &
Sviluppo di un sistema di notifiche push \\
\hline
Sviluppo di una \gloss{direttiva} per visualizzare un burn-down chart per uno
  sprint &
Collegamento tra area Scrum e Forum \\
\hline
\end{tabular}
\captionof{table}{Obiettivi produttivi}
\label{tab:obiettivi-produttivi}

%**************************************************************
\subsection{Il progetto visto dallo stagista}

L'attività di stage è inserita nel piano di studi di un Corso di Laurea.

Proprio per questo motivo, il Piano di Lavoro è stato pensato affinchè io
riuscissi a guadagnare qualcosa dallo stage: sono stati infatti individuati
degli obiettivi (formativi) minimi e massimi. \\

\begin{tabular}{| p{6cm} | p{6cm} |}

\hline
\textbf{Obiettivi minimi} & \textbf{Obiettivi massimi} \\
\hline
Ottenimento di una conoscenza profonda di Scrum &
Capacità di adattare i principi di Scrum ad esigenze diverse dallo sviluppo
  software \\
\hline
Acquisizione delle competenze necessarie per sviluppare le funzionalità
  previste negli obiettivi produttivi &
\mbox{} \\
\hline
\end{tabular}
\captionof{table}{Obiettivi formativi}
\label{tab:obiettivi-formativi}

\paragraph{La scelta} \mbox{} \\

Prima dello stage la mia conoscenza sull'azienda era limitata e di Jgalileo
sapevo solamente pochi dettagli.
L'incontro avvenuto a
Stage-IT\footnote{\url{http://informatica.math.unipd.it/laurea/stageit.html}}
mi ha permesso di conoscere meglio la realtà aziendale e di richiedere un
progetto che potesse includere lo studio di Scrum, siccome i metodi agili
costituiscono un campo di studio molto coinvolgente per me.

Qualche decennio fa i metodi con impostazione classica trovavano spazio
e efficacia di applicazione, siccome il software era potenzialmente
assimilabile ad un'attività di tipo industriale, dove requisiti e architettura
vengono stabiliti inizialmente e mantenuti per mesi, fino ad una nuova analisi
di mercato.

Con l'avvento della globalizzazione non sempre è possibile applicare questi
metodi, vista la mutevolezza del mondo esterno e dei suoi bisogni. Per questo
motivo, ritengo che lo studio e l'applicazione di Scrum e altri metodi
agili possa essere determinante per evitare il fallimento di progetti
software.

Oltre alla tematica principale dello stage, altri motivi che mi hanno portato
a questa scelta sono:

\begin{itemize}
\item Conoscere parte del mondo dei software gestionali;
\item Lavorare in un ambiente giovane, con molte persone al suo interno;
\item Poter contribuire direttamente all'efficienza dei processi aziendali;
\item Poter lavorare su una web application con tecnologie a me nuove.
\end{itemize}

%**************************************************************
\section{Come l'azienda usa gli stage}

Sanmarco Informatica, come detto nel primo capitolo, fornisce diverse
opportunità di stage, a seconda della provenienza dei tirocinanti (scuole
superiori, università o altro).

È di particolare interesse comprendere come queste attività vengano adoperate
dall'azienda per formare tali risorse o ottenere benefici dal lavoro di queste
nel breve periodo e, nel secondo caso, il valore aggiunto che questi benefici
portano.

%**************************************************************
\subsection{Visione dell'attività stage}

\paragraph{Formazione} \mbox{} \\

Mentre per gli stage di diversa natura (e.g. scuole superiori o iniziative
indipendenti da istituzioni scolastiche) le risorse sono sottoposte a dei
corsi di formazione che spesso partono dai fondamenti della programmazione ad
oggetti, nel caso degli stage per studenti al termine degli studi di
Informatica non è previsto nessun corso iniziale da parte di personale
aziendale.

Tuttavia, non avendo conoscenze approfondite su tecnologie e metodologie usate
dal team in cui sono stato inserito, la prima settimana è stata dedicata allo
studio e all'approfondimento (come indicato in tabella
\ref{tab:piano-di-lavoro}) come preparazione per le successive settimane.

\paragraph{Benefici derivanti dallo stage} \mbox{} \\

La forza-lavoro di cui un'azienda dispone grazie alle attività di stage può
essere utilizzata in diversi modi.

Per quanto riguarda Sanmarco Informatica, vi sono principalmente tre visioni
sull'attività di stage: una volta alla formazione, una alla prototipazione e
una terza dedicata allo sviluppo.

Come detto precedentemente, gli stage fortemente orientati alla formazione
sono riservati a coloro che non dispongono di un ricco background informatico,
in modo tale da permettere loro di acquisire le competenze necessarie per
poter lavorare in una software house che sviluppa \gloss{erp}.

Il secondo modo in cui l'azienda usa gli stage è per prototipazione: in questo
caso lo stagista prova a percorrere strade potenzialmente utili al resto del
team, sebbene non vi sia la garanzia che il prototipo prodotto venga
effettivamente integrato in un secondo momento.

Questa alternativa è dovuta al fatto che molte volte il team già inserito in
azienda non ha tempo per sperimentare nuove soluzioni, nonostante queste
potrebbero portare notevoli benefici.

Infine, Sanmarco Informatica può utilizzare gli stage per lo sviluppo di nuove
funzionalità in prodotti già esistenti, come nel mio caso.

Durante la mia permanenza in azienda, ho potuto infatti vedere JPA evolversi
grazie al mio contributo.

Un altro vantaggio derivante da questo tipo di stage è il carico di
responsabilità attribuitami: eventuali difetti introdotti nel prodotto
avrebbero potuto danneggiare l'intero team in tempi molto brevi. Questo
fattore mi ha permesso di percepire l'impatto che eventuali mie azioni
sbagliate avrebbero potuto avere e allo stesso tempo mi ha permesso di tenere
alta la concentrazione.

%**************************************************************
\subsection{Integrazione nel team}
Durante l'attività di stage ho lavorato con il team che sviluppa JPA, a capo
del quale vi è Alex Beggiato (il mio tutor aziendale).

Questo team è composto da elementi giovani, alcuni di questi laureati da poco
in Informatica o lauree affini a questa. Questo fattore e la disponibilità
mostrata nei miei confronti ha reso molto semplice l'inserimento nel team.

Le funzionalità di JPA che ho sviluppato attualmente hanno come
\emph{stakeholders} il resto del team ed è stato determinante poter lavorare
al loro fianco e avere un riscontro continuo per capire quanto ciò che stavo
realizzando rispondesse correttamente alle loro necessità.

%**************************************************************
\subsection{Gestione degli obiettivi di stage}\label{sec:visione-gestione-ob}

Come detto in sezione \ref{sec:strat-prog}, prima dello stage è stato fissato
un Piano di Lavoro in quale erano fissati gli obiettivi produttivi.

\paragraph{Granularità} \mbox{} \\

Gli obiettivi individuati erano stati definiti in modo generico, posticipando
la loro specifica precisa al momento in cui sarebbe cominciato il lavoro per
raggiungerli.

Al momento della stesura del Piano di Lavoro erano note le necessità e le
principali lacune dell'area Scrum di JPA, ma c'era bisogno di tempo affinchè
il team (me compreso) utilizzasse questa parte del prodotto e decidesse quale
fosse la migliore evoluzione.

Per questo motivo, i requisiti e la documentazione da produrre per ciascun
incremento è sempre stata decisa all'inizio di ogni sprint.

\paragraph{Flessibilità} \mbox{} \\

Gli obiettivi di stage erano flessibili sia per quanto riguardava la loro
granularità che per il fatto di poter evolvere nel caso in cui fossero stati
completati anticipatamente.

Questo ha permesso di realizzare ulteriori funzionalità una volta completate
quelle inizialmente previste dal Piano di Lavoro.

%**************************************************************
\section{Vincoli}

%**************************************************************
\subsection{Vincoli tecnologici}

Per lo sviluppo delle funzionalità elencate nel Piano di Lavoro ho dovuto
utilizzare le tecnologie elencate in sezione \ref{sec:azienda-tecnologie}.

Oltre a queste, nel corso dello stage sono stati imposti ulteriori vincoli
tecnologici in base alle funzionalità che vi era la necessità di implementare.

\paragraph{Google Chart} \mbox{} \\

Per lo sviluppo degli strumenti di \emph{Inspection} ho dovuto utilizzare le
Google Chart \gloss{api}\footnote{Oltre a questa libreria è stata analizzata
la possibilità di usare Chart.js o D3.js. Tuttavia, non vi erano significativi
vantaggi nell'utilizzare questi rispetto alle Google Chart \gloss{api}},
poichè il resto del team aveva già cominciato a sviluppare dei prototipi e
acquisito familiarità con queste.

Questa libreria presenta i seguenti punti di forza:

\begin{itemize}
\item Numerosità elevata dei tipi di grafico, offrendo la possibilità di
  combinarli;
\item Utilizzo di HTML5 e supporto per diversi browser;
\item Possibilità di aggiornare il grafico senza ricaricare la pagina;
\item Grafici mostrati in formato \gloss{svg} (vettoriale), garantendo un'alta qualità
  dell'immagine anche su schermi ad alta risoluzione o con zoom;
\item Oltre a visualizzare l'immagine del grafico, sono disponibili degli
  utili comandi per ciascun tipo di grafico;
\item Il loro uso è gratuito.
\end{itemize}

Il loro uso però ha comportato le seguenti difficoltà:

\begin{itemize}
\item Complessità nell'integrarli in una web application scritta in AngularJS:
  per l'integrazione è stata usata la \gloss{direttiva}
  \texttt{angular-google-chart} v.0.0.11 che non copre completamente tutte le
  possibilità offerte dalle Google Chart \gloss{api}. Ad esempio, non tutti i
  tipi di grafico erano supportati e ho dovuto implementare alcune
  funzionalità che sarebbero state disponibili nelle \gloss{api} ``originali'';

\begin{figure}[H]%
\centering
\includegraphics[width=.5\columnwidth]{immagini/ang-goog-chart-logo}
\caption{Logo della \gloss{direttiva} \texttt{angular-google-chart}}
\source{\url{https://github.com/angular-google-chart/angular-google-chart/blob/gh-pages/images/logo/AGC-Logo.svg}}
\label{fig:logo-agc}%
\end{figure}

\item Struttura complessa per l'inserimento di dati: la ricca documentazione
  delle Google Chart \gloss{api} non fornisce uno schema per le proprietà
  degli oggetti rappresentati un grafico. \\
  La difficoltà incontrata consiste infatti nel dover dedurre da vari esempi
  sparsi nella documentazione il formato di dati corretto per un grafico,
  siccome le funzioni di base disponibili per l'aggiunta di righe e colonne
  non permettono di costruire grafici ricchi di dettagli.
\end{itemize}

\paragraph{Font Awesome Icons} \mbox{} \\

Nelle pagine web di JPA è stato utilizzato il set di icone \textbf{Font
Awesome Icons} v.4.1.0, che porta con sè diversi benefici:

\begin{itemize}
\item Si hanno a disposizione centinaia di icone con stile grafico uniforme;
\item Le icone sono in formato \gloss{svg}, perciò la qualità della loro
  visualizzazione rimane alta anche su schermi ad alta risoluzione o
  effettuando uno zoom su queste;
\item Non serve JavaScript per visualizzare le icone in questo set,
  alleggerendo il caricamento della pagina e la facilità di impiego;
\item È stato sviluppato originariamente per integrarsi con Bootstrap,
  \gloss{framework} grafico utilizzato per il \FREND{} di JPA;
\item Risulta facile modificare le icone presenti e fornire stili aggiuntivi
  con regole CSS;
\item Questo insieme di icone è ottimo anche dal punto di vista
  dell'accessibilità fornendo supporto agli screen reader;
\item Il suo uso è gratuito.
\end{itemize}

\begin{figure}[H]%
\centering
\includegraphics[width=.5\columnwidth]{immagini/logo-fa}
\caption{Logo di Font Awesome Icons}%
\source{\url{https://fortawesome.github.io/Font-Awesome/icons/}}
\label{fig:logo-fa}%
\end{figure}

%**************************************************************
\subsection{Vincoli metodologici}

\paragraph{Interazione con tutor aziendale} \mbox{}

Durante lo stage vi sono state tre modalità di dialogo con il tutor aziendale:
analisi, \emph{Daily Scrum} e chiarimenti.

L'analisi veniva effettuata ad ogni pianificazione ed era volta a definire e
disambiguare gli obiettivi del Piano di Lavoro. Tale attività aveva durata
variabile a seconda dello sforzo necessario a portare a termine tale obiettivo.

Il \emph{Daily Scrum} consisteva in un confronto quotidiano per fare il punto
della situazione ed evidenziare eventuali difficoltà incontrate.

Il resto delle interazioni con il tutor aziendale sono state perlopiù
chiarimenti, siccome ho svolto lo stage nello stesso open space dove lavorava
il team. Queste avvenivano principalmente per:

\begin{itemize}
\item Capire se la parte di incremento parzialmente sviluppata era corretta
  rispetto alle attese;
\item Chiedere informazioni riguardo norme o raccomandazioni per lo sviluppo;
\item Ricevere informazioni su variazioni o chiarimenti di requisiti.
\end{itemize}

\paragraph{Pianificazione dello sprint} \mbox{}

La pianificazione dello sprint (\emph{Sprint Planning}) avveniva
settimanalmente e serviva per definire più precisamente gli obiettivi da portare a termine e determinare i vari task da svolgere necessari per
completare un incremento.

Una volta terminata l'analisi, era mio compito:

\begin{enumerate}
\item Creare uno sprint in JPA per la settimana lavorativa;
\item Attivare tale sprint;
\item Creare i task necessari per riuscire a sviluppare l'incremento;
\item Inserire tali task nello sprint appena creato.
\end{enumerate}

\paragraph{Consegna dell'incremento di uno sprint} \mbox{}

Al termine dello sprint, le parti dell'incremento venivano consegnate con 
diverse modalità a seconda della loro natura:

\begin{itemize}
\item Le modifiche agli script per JPADb venivano integrate appena testate;
\item I documenti venivano consegnati in allegato alla mail di resoconto
  inviata al tutor interno;
\item Le modifiche al \BKEND{} venivano integrate al ramo principale di SVN;
\item Le modifiche al \FREND{} venivano consegnate personalmente al tutor
  aziendale affinchè fossero verificate più attentamente, per essere integrate
  in un secondo momento.
\end{itemize}

%**************************************************************
\subsection{Vincoli temporali}

Lo stage previsto al termine del corso di studi della laurea triennale in
Informatica ha durata compresa tra le 300 e le 320 ore. In questo caso lo
stage ha impiegato 320 ore, per un totale di 8 settimane lavorative.

Per la maggior parte dei 40 giorni (circa il 70\% di questi) ho lavorato tra i
30 e i 40 minuti in più al giorno, sia anticipando le entrate che posticipando
le uscite.

Questa scelta è completamente personale, dal momento che spesso per avviare il
computer e la macchina virtuale su cui lavoravo erano necessari tra i 10 e i
20 minuti. In questo modo, ho potuto impiegare pienamente un tempo leggermente
maggiore delle 320 ore di stage per raggiungere gli obiettivi previsti e
quelli aggiuntivi.

%**************************************************************
\section{Prospettive di fine stage}

Una volta terminato lo stage, diventa interessante sapere cosa avviene in
seguito.

\paragraph{Assunzione} \mbox{} \\

A seguito di un'attività formativa, l'azienda potrebbe decidere di confermare
le risorse in cui ha investito. Nel caso di Sanmarco Informatica, questa
ipotesi assume sfumature molto concrete:

\begin{itemize}
\item al termine dei corsi di formazione, mediamente un terzo o metà dei
  partecipanti trova occupazione presso l'azienda;
\item nel team in cui sono stato inserito, metà dei componenti erano stati
assunti dopo uno stage universitario come il mio.
\end{itemize}

\paragraph{Uso del prodotto} \mbox{} \\

Un altro aspetto, anticipato nei precedenti paragrafi, è come quanto prodotto
durante lo stage verrà utilizzato una volta conclusa l'attività.

Nel mio caso le funzionalità stanno venendo integrate completamente man mano
che vengono testate dal tutor aziendale ed alcune erano già presenti in JPA al
termine delle otto settimane.

Oltre a ciò, la documentazione prodotta potrà essere utilizzata per:

\begin{itemize}
\item testare quanto realizzato;
\item avere spunti su come evolvere JPA, per avvicinarlo maggiormente a Scrum
  o per estendere funzionalità introdotte durante lo stage;
\item per fornire istruzioni per l'utilizzo di JPA.
\end{itemize}
