% !TEX encoding = UTF-8
% !TEX TS-program = pdflatex
% !TEX root = ../tesi.tex
% !TEX spellcheck = it-IT

%**************************************************************
\chapter{Svolgimento dello stage}
\label{cap:progetto}
%**************************************************************

\intro{In questo capitolo verranno descritti i dettagli riguardanti
l'attività di stage.}\\

%**************************************************************
\section{Pianificazione}

Man mano che venivano definite, le attività pianificate di settimana in
settimana sono state inserite in JPA in sprint visibili solamente a me.

In questi sprint i task venivano creati da me e assegnati a me stesso,
utilizzando i tipi disponibili nel seguente modo:

\begin{itemize}
\item \textbf{Nuovo sviluppo:} documentazione
\item \textbf{Implementazione:} realizzazione di funzionalità
\item \textbf{Errore:} anomalie riportate dal tutor aziendale o dagli altri
  membri del team
\end{itemize}

Sebbene dalla quarta settimana fossero disponibili anche le checklist per
pianificare ed organizzare il proprio lavoro, non ho mai avuto necessità di
utilizzarle siccome gli sprint da me utilizzati erano personali. A causa
di ciò, anche decomponendo i task più complessi in un insieme di task, la
visualizzazione dello sprint rimaneva sempre chiara.

La pianificazione effettuata durante l'attività di stage ha raffinato gli
obiettivi ed il Piano di Lavoro inizialmente individuati.

Dal momento che durante lo stage alcune parti del lavoro previsto sono state
completate anticipatamente, sono stati aggiunti nuovi obiettivi produttivi ad
attività in corso.

Le deviazioni rispetto al Piano di Lavoro (tabella \ref{tab:piano-di-lavoro})
delle otto settimane lavorative di stage sono riportate in tabella
\ref{tab:deviazioni}.

Si può in particolare notare che la maggior parte del lavoro svolto nelle
settimane 5, 7 e 8 non era prevista inizialmente. Questi incrementi sono stati
concordati con il tutor aziendale durante la mia permanenza a Sanmarco
Informatica, analizzando ulteriori necessità del team.

\begin{tabular}[t]{| c | p{10cm} |}

\hline
\textbf{Settimana} & \textbf{Lavoro previsto} \\
\hline
1 &
Sviluppo di un layout alternativo per un modulo (non previsto). \\
\hline
2 &
Mancato sviluppo di creazione checklist da template (obbligatorio) e
  importazione template in checklist (opzionale). \\
\hline
3 &
Sviluppo creazione checklist da template (in ritardo dalla settimana 2). \\
\hline
4 &
Completamento modulo burn-down chart, documentazione esclusa. \\
\hline
5 &
Importazione di template in checklist (opzionale, in ritardo dalla settimana
  2). Completamento documentazione burn-down chart. Sviluppo avvio istanza da
  task Scrum (in anticipo, previsto per settimana 6). Sviluppo di ulteriori
  strumenti di Inspection: burn-up chart, pie chart, cumulative flow diagram e
  una variante del burn-down chart (inizialmente non previsti). \\
\hline
6 &
Analisi, sviluppo e documentazione sul collegamento tra modulo scrum e modulo
  di notifica (anticipato dalla settimana 7). \\
\hline
7 &
Analisi, sviluppo e documentazione sul collegamento tra modulo scrum e
  modulo forum (anticipato dalla settimana 8). Sviluppo di \gloss{api} per
  JPAUtil (inizialmente non previste). Implementazione di un nuovo tipo di
  notifica utilizzando un Telegram Bot. \\
\hline
8 &
Sviluppo di un sistema di gestione del collegamento tra il proprio utente e il
  proprio numero telefonico/account su Telegram. Sviluppo di un modulo di
  istruzioni per la creazione di un proprio Telegram Bot. Prototipazione di
  invio di documenti. Incremento in termini di sicurezza dell'accoppiamento tra
  un utente e il Telegram Bot. Studio della realtà aziendale e inizio stesura
  della relazione finale. \\
\hline
\end{tabular}
\captionof{table}{Deviazioni rispetto al Piano di Lavoro}
\label{tab:deviazioni}

\paragraph{Strumenti di \emph{Inspection}} \mbox{} \\

Un burn-down chart può non essere l'unico strumento di \emph{Inspection} utile
per un gruppo di lavoro, poichè non riesce a mostrare informazioni riguardo a
diversi indicatori di performance:

\begin{itemize}
\item Un \textbf{burn-up chart} mostra quante attività vengono aggiunte man
  mano che lo sprint avanza, mettendo in evidenza eventuali lacune
  nell'attività di pianificazione;
\item Un \textbf{cumulative flow diagram} mostra i livelli in cui risiedono
  maggiormente i task, trovando eventuali colli di bottiglia nei processi di
  sviluppo aziendale e permettendo di ricavare utili informazioni come il
  \gls{lead-time}.
\end{itemize}

\paragraph{Notifiche su Telegram} \mbox{} \\

Le funzionalità di notifica previste dal Piano di Lavoro riguardavano l'invio
di mail interne tramite un modulo di JPA dedicato a tale scopo.

Tuttavia, tale parte dell'applicazione è ancora in via di sviluppo e presenta
delle limitazioni come il fatto di non poter utilizzata al di fuori di JPA.

Per questo motivo, durante lo stage è stato concordato lo sviluppo di un nuovo
tipo di notifiche, trasmesse su Telegram.

Telegram è un programma di messaggistica istantanea completamente
gratuito\footnote{a differenza di altri software, anche le \gloss{api}
disponibili sono gratuite} che ha nella velocità e nella sicurezza i suoi
punti di forza. Questa applicazione è disponibile sia su dispositivi mobili
che su computer ed è permesso l'invio di qualsiasi tipo di file.

Dal 24 giugno di quest'anno Telegram ha reso disponibile i \textbf{Telegram
Bot}, ovvero degli account che ricevono e inviano messaggi via software e non
per comando diretto di persone.

Grazie a questi Bot è quindi possibile automatizzare attività e, nel caso del
mio stage, inviare messaggi contenenti notifiche relative ad eventi su JPA.

%**************************************************************
\section{Norme e strumenti}

\subsection{Formazione delle conoscenze mancanti}

\subsection{Comunicazione tra membri del team}

\subsection{Sviluppo}

\section{Resoconto dell'attività di stage}

\subsection{Prima settimana -- formazione}

\subsection{Analisi}

\subsection{Progettazione}

\subsection{Codifica}

\subsection{Verifica ed integrazione}
