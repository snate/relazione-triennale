%**************************************************************
% Glossario
%**************************************************************
\renewcommand{\glossaryname}{Glossario}

\newglossaryentry{agilemanifesto}
{
    name={Agile Manifesto},
    text=Agile Manifesto,
    sort=agile manifesto,
    description={Pubblicazione in cui vengono espresse le fondamenta delle metodologie di sviluppo software agili}
}

\newglossaryentry{api}
{
    name={API},
    text=API,
    sort=api,
    description={(\emph{Application Programming Interface}) Con questo termine si indica una o più componenti software che offrono delle funzionalità, specificandone operazioni, input, output e tipi di queste. In questo modo, è più semplice realizzare un prodotto software facendo uso di questo insieme di strumenti di utilità}
}

\newglossaryentry{applicationserver}
{
    name={Application server},
    text=application server,
    sort=application server,
    description={\Gloss{framework} che rende più facile la creazione e l'esecuzione di web application}
}

\newglossaryentry{bpm}
{
    name={BPM},
    text=BPM,
    sort=bpm,
    description={(\emph{Business Process Management}) Insieme di attività necessarie per definire, modellare, eseguire, monitorare, ottimizzare ed integrare i processi aziendali per riuscire a migliorare questi lungo gli assi di efficacia (soddisfazione delle aspettative) ed efficienza (contenimento dei costi)}
}

\newglossaryentry{cerimonia}
{
    name={Cerimonia},
    text=cerimonia,
    plural=cerimonie,
    sort=cerimonia,
    description={Evento caratteristico della metodologia Scrum in cui alcuni degli \emph{stakeholders} interagiscono in un intervallo di tempo limitato per sincronizzarsi su un determinato aspetto riguardante lo stato del progetto}
}

\newglossaryentry{chiaveprimaria}
{
    name={Chiave primaria},
    text=chiave primaria,
    plural=chiavi primarie,
    sort=chiave primaria,
    description={Campo o insieme di campi di una tabella che identifica in modo univoco un elemento all'interno di questa}
}

\newglossaryentry{crud}
{
    name={CRUD},
    text=CRUD,
    sort=crud,
    description={(\emph{Create Read Update Delete}) Insieme di operazioni comuni alla maggior parte dei software che utilizzano un database: creazione, aggiornamento, lettura ed eliminazione di tabelle o righe di queste}
}

\newglossaryentry{direttiva}
{
    name={Direttiva},
    text=direttiva,
    plural=direttive,
    sort=direttiva,
    description={Funzione che, in AngularJS, viene applicata ad uno specifico elemento della pagina per aggiungere funzionalità aggiuntive a questo (validazione di input, gestione di eventi, etc.)}
}

\newglossaryentry{erp}
{
    name={ERP},
    text=ERP,
    sort=erp,
    description={(\emph{Entity Resource Planning}) Insieme di applicazioni integrate che un'azienda utilizza per gestire i propri processi}
}

\newglossaryentry{framework}
{
    name={Framework},
    text=\emph{framework},
    sort=framework,
    description={Architettura fortemente riusabile che offre dei benefici a costo di rispettare alcuni vincoli che l'uso di essa impone}
}

\newglossaryentry{helpdesk}
{
    name={Help desk},
    text=help desk,
    sort=help desk,
    description={Servizio aziendale che fornisce informazioni ed assistenza a clienti e utenti, interni o esterni, che hanno problemi nella gestione di un prodotto o servizio}
}

\newglossaryentry{ide}
{
    name={IDE},
    text=IDE,
    sort=ide,
    description={(\emph{Integrated Development Enviroment}) Applicazione il cui scopo è aumentare la produttività durante lo sviluppo software, fornendo funzionalità ulteriori (possono consistere in compilazione del codice, debugging, etc.) rispetto alla semplice modifica del testo di un file}
}

\newglossaryentry{jaxrs}
{
    name={JAX-RS},
    text=JAX-RS,
    sort=jaxrs,
    description={(\emph{Java API for RESTful Web Services}) Insieme di API per il linguaggio di programmazione Java che facilitano la realizzazione di servizi Web RESTful}
}

\newglossaryentry{lead-time}
{
    name={Lead time},
    text=\emph{lead time},
    sort=lead time,
    description={Intervallo temporale che scorre dall'inizio al termine dell'esecuzione di un processo. Questo indicatore è talvolta chiamato \textbf{tempo di risposta}}
}

\newglossaryentry{milestone}
{
    name={Milestone},
    text=milestone,
    sort=milestone,
    description={Momento in cui è previsto il conseguimento di obiettivi di importanza primaria per un progetto}
}

\newglossaryentry{mission}
{
    name={Mission},
    text=mission,
    sort=mission,
    description={Scopo per il quale esiste un'azienda ed è la motivazione che risiede dietro ogni decisione di questa}
}

\newglossaryentry{pull-model}
{
    name={Pull model},
    text=\emph{pull model},
    sort=pull model,
    description={Ottenimento di informazioni da parte del client che avviene
    tramite richiesta di questo al server. Tale modalità è opposta al
    \gls{push-model}}
}

\newglossaryentry{rdbms}
{
    name={RDBMS},
    text=RDBMS,
    sort=rdbms,
    description={(\emph{Relational DataBase Management System}) Sistemi per mezzo dei quali si può accedere a basi di dati strutturate per tabelle e relazioni tra queste}
}

\newglossaryentry{repository}
{
    name={Repository},
    text=repository,
    sort=repository,
    description={Archivio di dati informatico, in cui possono essere raccolti sia pacchetti software che altri tipi di dati}
}

\newglossaryentry{responsive}
{
    name={Responsive},
    text=\emph{responsive},
    sort=responsive,
    description={Capacità di un layout di riuscire ad adattarsi allo schermo o alla porzione di schermo su cui la pagina Web è visualizzata, ridimensionando o spostando in modo opportuno alcuni elementi all'interno della pagina}
}

\newglossaryentry{rest}
{
    name={REST},
    text=REST,
    sort=rest,
    description={(\emph{REpresentational State Transfer}) Architettura di rete ha le sue fondamenta nel concetto di \textbf{risorsa}, ovvero una fonte di informazioni raggiungibile tramite un indirizzo. Questo protocollo è \emph{client-server}, in quanto ciascuno dei due sistemi può evolvere indipendentemente dall'altro grazie ad un significativo disaccoppiamento tra le due parti di architettura}
}

\newglossaryentry{spa}
{
    name={SPA},
    text=SPA,
    sort=spa,
    description={(\emph{Single Page Application}) Applicazione web che risiede in un'unica pagina. I link per la navigazione all'interno del sito portano ad altre pagine che vengono caricate dinamicamente dentro o al posto della pagina corrente, fornendo un'esperienza utente più vicina a quella delle applicazioni Desktop che a quella dei siti Web tradizionali}
}

\newglossaryentry{svg}
{
    name={SVG},
    text=SVG,
    sort=svg,
    description={(\emph{Scalable Vectorial Graphics}) Formato utilizzato per rappresentare immagini bidimensionali, raccomandazione \gloss{w3c} dal 1999 e supportato dalla maggior parte dei browser. Ha come vantaggio principale il fatto di essere scalabile (ridimensionabile senza perdite di qualità), salvando le immagini in formato vettoriale e non come mappe di pixel}
}

\newglossaryentry{suite}
{
    name={Suite},
    text=suite,
    sort=suite,
    description={Insieme di prodotti software che sono affini tra loro per tipo di funzionalità offerte}
}

\newglossaryentry{teamviewer}
{
    name={Team Viewer},
    text=Team Viewer,
    sort=team viewer,
    description={Applicazione con la quale è possibile controllare remotamente un altro computer}
}

\newglossaryentry{uml}
{
    name={UML},
    text=UML,
    sort=uml,
    description={(\emph{Unified Modeling Language}) In ingegneria del software \emph{UML, Unified Modeling Language} (ing. linguaggio di modellazione unificato) è un linguaggio di modellazione e specifica basato sul paradigma object-oriented. L'\emph{UML} svolge un'importantissima funzione di ``lingua franca'' nella comunità della progettazione e programmazione a oggetti. Gran parte della letteratura di settore usa tale linguaggio per descrivere soluzioni analitiche e progettuali in modo sintetico e comprensibile a un vasto pubblico}
}

\newglossaryentry{vcs}
{
    name={VCS},
    text=VCS,
    sort=vcs,
    description={(\emph{Version Control System}) Software con il quale viene gestita l'evoluzione di un \gloss{repository} nel tempo. Grazie a questo strumento possono ad esempio essere sviluppate più versioni e varianti dello stesso prodotto su diversi rami di produzione oppure la correzione di errori e l'implementazione di nuove funzionalità può essere realizzata e testata a sufficienza su rami secondari prima di essere integrata nella versione stabile del prodotto.
    Ciascuno dei rami è composto da una sequenza di nodi, chiamati \emph{commit}. Le modifiche ad un certo \gloss{repository} sono salvate a struttura ad albero, quindi vi è lo stesso commit iniziale che funge da radice per tutti i possibili rami che tale sistema può contenere}
}

\newglossaryentry{w3c}
{
    name={W3C},
    text=W3C,
    sort=w3c,
    description={(\emph{World Wide Web Consortium}) Comunità internazionale che produce standard liberamente accessibili al fine di sostenere la crescita del Web al suo massimo potenziale}
}
