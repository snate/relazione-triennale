\documentclass[
article,
10pt, % Main document font size
oneside, % One page layout (no page indentation)
BCOR5mm, % Binding correction
]{scrartcl}
\usepackage[a4paper,margin=3cm]{geometry}
\usepackage[italian]{babel}
\usepackage[utf8]{inputenc}

\hyphenation{Fortran hy-phen-ation} % Specify custom hyphenation points in 
  %words with dashes where you would like hyphenation to occur, or
  %alternatively, don't put any dashes in a word to stop hyphenation altogether

\begin{document}

\title{Struttura della relazione finale}
\author{Sebastiano Valle}

\maketitle

\begin{abstract}
Questo documento riporta la struttura di capitoli e sezioni per la relazione
finale sull'attività di stage svolta da Sebastiano Valle presso l'azienda
Sanmarco Informatica.
\end{abstract}

\section{Realtà aziendale} % TODO: cambiare?

\paragraph{} Tale capitolo serve per descrivere l'azienda o, più in generale,
il dominio applicativo in cui si è svolto lo stage.

\subsection{L'azienda} % TODO: rendere più accattivante?

\paragraph{} Tale sezione introduce l'azienda, descrivendo la sua mission,
la collocazione delle sedi e l'evoluzione nel tempo dell'azienda e della sua
natura.

\subsection{Prodotti e servizi offerti}

\paragraph{} Tale sezione identifica i principali prodotti e servizi offerti
dall'azienda, soffermandosi brevemente a descrivere il prodotto le cui
funzionalità sono state estese durante l'attività di stage.

\subsection{Scrum aziendale}

\paragraph{} Tale sezione descrive l'organizzazione dei processi aziendali,
volta a produrre con regolarità degli incrementi potenzialmente rilasciabili ai
clienti.

\subsection{Tecnologie utilizzate}

\paragraph{} Tale sezione descrive le tecnologie utilizzate nello sviluppo del
prodotto esteso durante l'attività di stage, ovvero JPA.

\subsubsection{Database}

\paragraph{} In questa sottosezione vengono elencati i database supportati da
JPA, siccome la parte di back-end è pensata in modo tale da essere utilizzata
con diversi tipi di database. \\
Vengono inoltre indicate quali norme interne vengono adottate per la
nomenclatura di tabelle e campi.

\subsubsection{Back-end}

\paragraph{} In questa sottosezione viene spiegato quali strumenti e
tecnologie sono utilizzati per la parte di back-end di JPA.

Viene inoltre indicato:

\begin{itemize}
\item come viene effettuato il versionamento per il codice sorgente del
  back-end;
\item quali sono le norme interne adottate per la codifica.
\end{itemize}

\subsubsection{Front-end}

\paragraph{} In questa sottosezione viene spiegato quali strumenti e
tecnologie sono utilizzati per la parte di back-end di JPA.

Viene inoltre indicato:

\begin{itemize}
\item come viene effettuato il versionamento per il codice sorgente del
  front-end;
\item quali sono le norme interne adottate per la codifica.
\end{itemize}

\subsection{Innovazione}

\paragraph{} Tale sezione descrive il rapporto che l'azienda ha rispetto
all'innovazione e il modo di proporsi in tal senso ai clienti.

\end{document}
