\documentclass[
article,
10pt, % Main document font size
oneside, % One page layout (no page indentation)
BCOR5mm, % Binding correction
]{scrartcl}
\usepackage[a4paper,margin=3cm]{geometry}
\usepackage[italian]{babel}
\usepackage[utf8]{inputenc}

\hyphenation{Fortran hy-phen-ation} % Specify custom hyphenation points in 
  %words with dashes where you would like hyphenation to occur, or
  %alternatively, don't put any dashes in a word to stop hyphenation altogether

\begin{document}

\title{Struttura della relazione finale}
\author{Sebastiano Valle}

\maketitle

\begin{abstract}
Questo documento riporta la struttura di capitoli e sezioni per la relazione
finale sull'attività di stage svolta da Sebastiano Valle presso l'azienda
Sanmarco Informatica.
\end{abstract}

\section{Realtà aziendale} % TODO: cambiare?

\paragraph{} Tale capitolo serve per descrivere l'azienda o, più in generale,
il dominio applicativo in cui è collocato lo stage.

\subsection{L'azienda} % TODO: rendere più accattivante?

\paragraph{} Tale sezione introduce l'azienda, descrivendo la sua mission,
la collocazione delle sedi e l'evoluzione nel tempo dell'azienda e della sua
natura.

\subsection{Prodotti e servizi offerti}\label{sec:realta-prod}

\paragraph{} Tale sezione identifica i principali prodotti e servizi offerti
dall'azienda, soffermandosi brevemente a descrivere il prodotto (JPA) le cui
funzionalità sono state estese durante l'attività di stage.

\subsection{Scrum aziendale}

\paragraph{} Tale sezione descrive l'organizzazione dei processi aziendali,
volta a produrre con regolarità degli incrementi potenzialmente rilasciabili ai
clienti.

\subsection{Tecnologie utilizzate}

\paragraph{} Tale sezione descrive le tecnologie utilizzate nello sviluppo del
prodotto esteso durante l'attività di stage, ovvero JPA.

\subsubsection{Database}

\paragraph{} In questa sottosezione vengono elencati i database supportati da
JPA, siccome la parte di back-end è pensata in modo tale da essere utilizzata
con diversi tipi di database. \\
Vengono inoltre indicate quali norme interne vengono adottate per la
nomenclatura di tabelle e campi.

\subsubsection{Back-end}

\paragraph{} In questa sottosezione viene spiegato quali strumenti e
tecnologie sono utilizzati per la parte di back-end di JPA.

Viene inoltre indicato:

\begin{itemize}
\item come viene effettuato il versionamento per il codice sorgente del
  back-end;
\item quali sono le norme interne adottate per la codifica.
\end{itemize}

\subsubsection{Front-end}

\paragraph{} In questa sottosezione viene spiegato quali strumenti e
tecnologie sono utilizzati per la parte di back-end di JPA.

Viene inoltre indicato:

\begin{itemize}
\item come viene effettuato il versionamento per il codice sorgente del
  front-end;
\item quali sono le norme interne adottate per la codifica.
\end{itemize}

\subsection{Innovazione}

\paragraph{} Tale sezione descrive il rapporto che l'azienda ha rispetto
all'innovazione e il modo di proporsi in tal senso ai clienti.



\section{Strategie aziendali} % TODO: cambiare al singolare?

\paragraph{} Tale capitolo serve per descrivere come l'azienda percepisce le
attività di stage e come le gestisce, parlando in particolare dello stage a
cui la relazione finale fa riferimento.

\subsection{Il progetto}

\paragraph{} Tale sezione serve per spiegare in cosa consisteva il progetto
di stage.

\subsubsection{Il progetto visto dall'azienda}

\paragraph{} In questa parte della sezione viene spiegato come l'azienda
voleva estendere JPA (illustrato in \ref{sec:realta-prod}) e quali vantaggi
concreti ne poteva ricavare.

\subsubsection{Il progetto visto dallo stagista}

\paragraph{} In questa parte della sezione viene spiegato quali vantaggi lo
stagista può trarre dall'attività di stage, le aspettative iniziali e le
motivazioni della scelta.

\subsection{Visione dell'attività stage}

\paragraph{} Tale sezione descrive come l'azienda utilizza gli stage previsti
dal corso di Informatica dell'Università degli Studi di Padova.

\subsection{Come l'azienda usa gli stage}

\paragraph{} Tale sezione descrive come l'azienda usa gli stage, ovvero quali
responsabilità affidano allo stagista e come vengono gestiti gli obiettivi
produttivi.

\subsubsection{Integrazione nel team}

\paragraph{} In questa parte della sezione viene descritto l'accoglienza e il
supporto che il team ha riservato allo stagista fin dall'inizio dello stage.

\subsubsection{Gestione degli obiettivi di stage}

\paragraph{} In questa parte della sezione viene descritto come l'azienda
agisce rispetto alla pianificazione del lavoro all'inizio e durante lo stage:

\begin{itemize}
\item analizzando la granularità della pianificazione;
\item analizzando la flessibilità utilizzata per gli obiettivi produttivi.
\end{itemize}

\subsection{Vincoli}

\paragraph{} Tale sezione descrive quali vincoli sono stati imposti allo
stagista per portare a termine gli obiettivi prefissi.

\subsubsection{Vincoli tecnologici}

\paragraph{} In questa parte della sezione vengono discussi gli strumenti e le
tecnologie imposte per conseguire gli obiettivi.

\subsubsection{Vincoli metodologici}

\paragraph{} In questa parte della sezione vengono descritte:

\begin{itemize}
\item le modalità di interazione tra stagista e tutor;
\item l'uso di JPA per organizzare le proprie attività;
\item le modalità di consegna degli incrementi (documentali e non) prodotti.
\end{itemize}

\subsubsection{Vincoli temporali}

\paragraph{} In questa parte della sezione vengono discusse eventuali
deviazioni rispetto alle ore totali e all'orario di lavoro prefissato a inizio
stage.

\subsection{Prospettive di fine stage}

\paragraph{} Tale sezione racconta quali sono le opportunità che l'azienda
riserva allo stagista nel caso in cui l'esperienza di stage venga valutata
positivamente dal tutor aziendale.

\end{document}
