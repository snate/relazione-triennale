\documentclass[
article,
10pt, % Main document font size
oneside, % One page layout (no page indentation)
BCOR5mm, % Binding correction
]{scrartcl}
\usepackage[a4paper,margin=3cm]{geometry}
\usepackage[italian]{babel}
\usepackage[utf8]{inputenc}

\hyphenation{Fortran hy-phen-ation} % Specify custom hyphenation points in 
  %words with dashes where you would like hyphenation to occur, or
  %alternatively, don't put any dashes in a word to stop hyphenation altogether

\begin{document}

\title{Struttura della relazione finale}
\author{Sebastiano Valle}

\maketitle

\begin{abstract}
Questo documento riporta la struttura di capitoli e sezioni per la relazione
finale sull'attività di stage svolta da Sebastiano Valle presso l'azienda
Sanmarco Informatica.
\end{abstract}

\section{Realtà aziendale} % TODO: cambiare?

\paragraph{} Tale capitolo serve per descrivere l'azienda o, più in generale,
il dominio applicativo in cui è collocato lo stage.

\subsection{L'azienda} % TODO: rendere più accattivante?

\paragraph{} Tale sezione introduce l'azienda, descrivendo la sua mission,
la collocazione delle sedi e l'evoluzione nel tempo dell'azienda e della sua
natura.

\subsection{Prodotti e servizi offerti}\label{sec:realta-prod}

\paragraph{} Tale sezione identifica i principali prodotti e servizi offerti
dall'azienda, soffermandosi brevemente a descrivere il prodotto (JPA) le cui
funzionalità sono state estese durante l'attività di stage.

\subsection{Scrum aziendale}

\paragraph{} Tale sezione descrive l'organizzazione dei processi aziendali,
volta a produrre con regolarità degli incrementi potenzialmente rilasciabili ai
clienti.

\subsection{Tecnologie utilizzate}

\paragraph{} Tale sezione descrive le tecnologie utilizzate nello sviluppo del
prodotto esteso durante l'attività di stage, ovvero JPA.

\subsubsection{Database}

\paragraph{} In questa sottosezione vengono elencati i database supportati da
JPA, siccome la parte di back-end è pensata in modo tale da essere utilizzata
con diversi tipi di database. \\
Vengono inoltre indicate quali norme interne vengono adottate per la
nomenclatura di tabelle e campi.

\subsubsection{Back-end}

\paragraph{} In questa sottosezione viene spiegato quali strumenti e
tecnologie sono utilizzati per la parte di back-end di JPA.

Viene inoltre indicato:

\begin{itemize}
\item come viene effettuato il versionamento per il codice sorgente del
  back-end;
\item quali sono le norme interne adottate per la codifica.
\end{itemize}

\subsubsection{Front-end}

\paragraph{} In questa sottosezione viene spiegato quali strumenti e
tecnologie sono utilizzati per la parte di back-end di JPA.

Viene inoltre indicato:

\begin{itemize}
\item come viene effettuato il versionamento per il codice sorgente del
  front-end;
\item quali sono le norme interne adottate per la codifica.
\end{itemize}

\subsection{Innovazione}

\paragraph{} Tale sezione descrive il rapporto che l'azienda ha rispetto
all'innovazione e il modo di proporsi in tal senso ai clienti.



\section{Strategie aziendali} % TODO: cambiare al singolare?

\paragraph{} Tale capitolo serve per descrivere come l'azienda percepisce le
attività di stage e come le gestisce, parlando in particolare dello stage a
cui la relazione finale fa riferimento.

\subsection{Il progetto}

\paragraph{} Tale sezione serve per spiegare in cosa consisteva il progetto
di stage.

\subsubsection{Il progetto visto dall'azienda}

\paragraph{} In questa parte della sezione viene spiegato come l'azienda
voleva estendere JPA (illustrato in \ref{sec:realta-prod}) e quali vantaggi
concreti ne poteva ricavare.

\subsubsection{Il progetto visto dallo stagista}

\paragraph{} In questa parte della sezione viene spiegato quali vantaggi lo
stagista può trarre dall'attività di stage, le aspettative iniziali e le
motivazioni della scelta.

\subsection{Visione dell'attività stage}

\paragraph{} Tale sezione descrive come l'azienda utilizza gli stage previsti
dal corso di Informatica dell'Università degli Studi di Padova.

\subsection{Come l'azienda usa gli stage}

\paragraph{} Tale sezione descrive come l'azienda usa gli stage, ovvero quali
responsabilità affidano allo stagista e come vengono gestiti gli obiettivi
produttivi.

\subsubsection{Integrazione nel team}

\paragraph{} In questa parte della sezione viene descritto l'accoglienza e il
supporto che il team ha riservato allo stagista fin dall'inizio dello stage.

\subsubsection{Gestione degli obiettivi di stage}

\paragraph{} In questa parte della sezione viene descritto come l'azienda
agisce rispetto alla pianificazione del lavoro all'inizio e durante lo stage:

\begin{itemize}
\item analizzando la granularità della pianificazione;
\item analizzando la flessibilità utilizzata per gli obiettivi produttivi.
\end{itemize}

\subsection{Vincoli}

\paragraph{} Tale sezione descrive quali vincoli sono stati imposti allo
stagista per portare a termine gli obiettivi prefissi.

\subsubsection{Vincoli tecnologici}

\paragraph{} In questa parte della sezione vengono discussi gli strumenti e le
tecnologie imposte per conseguire gli obiettivi.

\subsubsection{Vincoli metodologici}

\paragraph{} In questa parte della sezione vengono descritte:

\begin{itemize}
\item le modalità di interazione tra stagista e tutor;
\item l'uso di JPA per organizzare le proprie attività;
\item le modalità di consegna degli incrementi (documentali e non) prodotti.
\end{itemize}

\subsubsection{Vincoli temporali}

\paragraph{} In questa parte della sezione vengono discusse eventuali
deviazioni rispetto alle ore totali e all'orario di lavoro prefissato a inizio
stage.

\subsection{Prospettive di fine stage}

\paragraph{} Tale sezione racconta quali sono le opportunità che l'azienda
riserva allo stagista nel caso in cui l'esperienza di stage venga valutata
positivamente dal tutor aziendale.



\section{Svolgimento dello stage} % TODO: titolo più accattivante?

\paragraph{} In questo capitolo verranno descritti i dettagli riguardanti
l'attività di stage.

\subsection{Pianificazione}
\paragraph{} Tale sezione descrive come, settimana per settimana, veniva
pianificato il lavoro.

\subsection{Norme e strumenti}
\paragraph{} Tale sezione descrive strumenti e norme adottati durante
l'attività di stage.

\subsubsection{Formazione delle conoscenze mancanti}
\paragraph{} In questa parte della sezione viene descritto come lo stagista ha
sopperito alle conoscenze inizialmente non possedute, sia riguardo le
tecnologie utilizzate che la metodologia Scrum;

\subsubsection{Comunicazione tra membri del team}
\paragraph{} In questa parte della sezione vengono descritte le modalità di
comunicazione tra i membri del team.

\subsubsection{Sviluppo}
\paragraph{} In questa parte della sezione viene descritto come lo sviluppo è
stato disciplinato per quanto riguarda:
\begin{itemize}
\item Analisi;
\item Progettazione;
\item Codifica (Database, Back-end, Front-end);
\item Documentazione.
\end{itemize}

\subsection{Resoconto dell'attività di stage}
\paragraph{} In questa sezione vengono illustrate le parti salienti delle otto
settimane di stage.

\subsubsection{Prima settimana -- formazione}
\paragraph{} Breve descrizione della prima settimana di stage (apprendimento
tecnologie nuove e studio della metodologia Scrum).

\subsubsection{Analisi}
\paragraph{} Requisiti individuati per gli incrementi di maggior rilievo
(collegamento di checklist a task scrum, burndown chart e altri grafici
prodotti).

\subsubsection{Progettazione}
\paragraph{} In questa parte della sezione viene descritta l'architettura
(back-end e front-end) pre-esistente e mostrando uno degli incrementi
prodotti\footnote{Ciò è dovuto al fatto che l'architettura pre-esistente era
facilmente estensibile e permetteva aggiunte facili e solitamente di
dimensione contenuta.}.

\subsubsection{Codifica}
\paragraph{} In questa parte della sezione vengono descritte quali procedure
per la codifica sono state seguite, le difficoltà incontrate, l'aiuto fornito
dal team.

\subsubsection{Verifica ed integrazione}
\paragraph{} Verifica degli incrementi prodotti e integrazione degli stessi
nell'applicazione utilizzata dal team.



\section{Valutazione retrospettiva} % TODO: qualcosa di più originale?

\subsection{Resoconto sugli obiettivi produttivi}
\paragraph{} In questa sezione vengono confrontati gli obiettivi produttivi
inizialmente previsti rispetto agli obiettivi effettivamente raggiunti.

Oltre a questo, vengono brevemente chiarite eventuali aggiunte rispetto agli
obiettivi iniziali, valutando se ad inizio stage potevano essere già previste
e il peso che hanno avuto sull'intera attività.

\subsection{Resoconto sugli obiettivi formativi}
\paragraph{} In questa sezione vengono confrontate le abilità possedute dallo
stagista ad inizio stage rispetto a quelle a fine stage.

In particolare, vengono espresse:

\begin{itemize}
\item delle valutazioni di carattere personale riguardo alla facilità di
  lavorare con un team disponibile e sempre pronto ad aiutare il compagno;
\item delle valutazioni retrospettive di quanto lo stagista ha appreso sullo
  sviluppo di SPA (Single Page Application) con HTML5 e AngularJS;
\item delle valutazioni retrospettive di quanto lo stagista ha appreso sullo
  sviluppo di layout per pagine web (CSS3 e Bootstrap);
\item delle valutazioni retrospettive di quanto lo stagista ha appreso sullo
  sviluppo della parte di back-end;
\item delle valutazioni retrospettive riguardo l'interrogazione di database
  tramite comandi SQL;
\item delle valutazioni retrospettive riguardo la conoscenza di Scrum.
\end{itemize}

\subsection{Distanza formativa tra corso di studi e realtà aziendale}
\paragraph{} In questa sezione vengono confrontate le nozioni insegnate
all'università e la differenza tra queste e il mondo lavorativo.

In particolare, vengono discussi:

\begin{itemize}
\item l'importanza di testare la correttezza dei propri programmi;
\item l'uso di diagrammi UML;
\item l'uso di design pattern, e come molti di questi vengano nativamente
  implementati nei framework;
\item l'uso di strumenti e procedure per la qualità;
\item la poca attenzione dei corsi verso alcune tecnologie emergenti,
  spesso utilizzate largamente da startup innovative e talvolta anche da
  aziende consolidate per restare aggiornate e non perdere fette di mercato;
\item l'apporto fornito dai vari corsi (Data Mining, Ingegneria del Software,
  Programmazione ad Oggetti, Programmazione Concorrente e Distribuita,
  Tecnologie Web);
\item la mancanza di iniziative nei vari corsi in cui studenti e professori
  possono collaborare come un vero team in azienda;
\item l'importanza di progetti didattici consistenti rispetto a progetti
  didattici che sono più di tipo \emph{one-off} e l'importanza che in generale
  hanno i progetti didattici.
\end{itemize}

\end{document}
